\documentclass[a4paper,10pt]{article}
\usepackage[utf8x]{inputenc}
\usepackage{listings}
\usepackage{a4wide}
\usepackage{tabularx}
\usepackage{booktabs}

\lstloadlanguages{Bash}
\lstset{basicstyle=\small\ttfamily, escapeinside={(*@}{@*)},
showstringspaces=false, breaklines=true, breakatwhitespace=true,
tabsize=4, language=Bash, numberfirstline=true,
commentstyle=\itshape\color{CommentGreen},
stringstyle=\color{DataTypeBlue}}
\newcommand{\ilcode}[1]{\lstinline|#1|}

\title{RTC:Joy User Guide}
\author{Geoffrey Biggs}

\begin{document}
\maketitle

\section{Introduction}
\label{sec:intro}

RTC:Joy is an RT-Component for joysticks under UNIX-based operating systems. It
supports the standard joystick interface exposed through device nodes such as
\verb|/dev/js*|.

This software is developed at the National Institute of Advanced Industrial
Science and Technology. Approval number H23PRO-1301. This software is licensed
under the Lesser General Public License. See COPYING and COPYING.LESSER in the
source.

\section{Requirements}
\label{sec:requirements}

RTC:Joy requires the C++ version of OpenRTM-aist-1.0.0.

RTC:Joy uses the CMake build system\footnote{http://www.cmake.org/}. You will
need at least version 2.6 to be able to build the component.

\section{Installation}
\label{sec:installation}

\subsection{Binary}

Users of Windows can install the component using the binary installer. This
will install the component and all its necessary dependencies. It is the
recommended method of installation in Windows.

\begin{enumerate}
  \item Download the installer from the website.
  \item Double-click the executable file to begin installation.
  \item Follow the instructions to install the component.
  \item You may need to restart your computer for environment variable changes
  to take effect before using the component.
\end{enumerate}

The component can be launched by double-clicking the \verb|rtcjoy_standalone|
executable. The \verb|rtcjoy| library is available for loading into a manager,
using the initialisation function \verb|rtc_init|.

\subsection{From source}

Follow these steps to install :

\begin{enumerate}
  \item Download the source, either from the repository or a source archive,
  and extract it somewhere.

  \verb|tar -xvzf rtc.tar.gz|
  \item Change to the directory containing the extracted source.

  \verb|cd rtc.0.0|
  \item Create a directory called ``build'':

  \verb|mkdir build|
  \item Change to that directory.

  \verb|cd build|
  \item Run cmake or cmake-gui.

  \verb|cmake ../|
  \item If no errors occurred, run make.

  \verb|make|
  \item Finally, install the component. Ensure the necessary permissions to
  install into the chosen prefix are available.

  \verb|make install|
  \item The install destination can be changed by executing ccmake and changing
  the variable \verb|CMAKE_INSTALL_PREFIX|.

  \verb|ccmake ../|
\end{enumerate}

The component is now ready for use. See the next section for instructions on
configuring the component.

RTC:Joy can be launched in stand-alone mode by executing the
\verb|rtcjoy_standalone| executable (installed into \verb|${prefix}/bin|).
Alternatively, \verb|librtcjoy.so| can be loaded into a manager, using the
initialisation function \verb|rtc_init|. This shared object can be found in
\verb|${prefix}/lib| or \verb|${prefix}/lib64|.


\section{Configuration}
\label{sec:configuration}

The available configuration parameters are described in
Table~\ref{tab:config_params}.

\begin{table}[t]
  \centering
  \begin{tabularx}{\columnwidth}{lX}
    \toprule
    Parameter & Effect \\
    \midrule
    device & Path of the joystick device node. \\
    x\_axis & Axis index to use for the X axis. \\
    y\_axis & Axis index to use for the Y axis. \\
    scale\_v & Scale factor to convert Y position into velocity in m/s. \\
    scale\_a & Scale factor to convert X position into velocity in rad/s. \\
    \bottomrule
  \end{tabularx}
  \caption{Available configuration parameters.}
  \label{tab:config_params}
\end{table}

\section{Ports}
\label{sec:port}

The ports provided by the component are described in Table~\ref{tab:ports}.

\begin{table}[t]
  \centering
  \begin{tabularx}{\columnwidth}{lllX}
    \toprule
    Name & Type & Data type & Purpose \\
    \midrule
    axes & OutPort & RTCJoyTypes::JSAxis & Axis values, normalised to between -1.0 and 1.0. \\
    buttons & OutPort & RTCJoyTypes::JSBtn & Button press and release events. \\
    xy & OutPort & RTC::TimedPoint2D & X/Y position. \\
    va & OutPort & RTC::TimedVelocity2D & Forward and angular velocity based on xy. \\
    \bottomrule
  \end{tabularx}
  \caption{Available ports.}
  \label{tab:ports}
\end{table}

\section{Examples}
\label{sec:examples}

An example configuration file is provided in the
\verb|${prefix}/share/rtc/examples/conf/| directory.

% \section{Changelog}


\end{document}
